\#Простой растровый графический редактор \paragraph*{1. Наименование}

Простой растровый графический редактор “r\+Astro”

\paragraph*{2. Назначение}

Назначение ПО – создание (и обработка?) растровых изображений, сохранение изображений в формате J\+P\+E\+G. ПО разрабатывается с целью получения навыков программирования и работы с проектами по разработке ПО.

\paragraph*{3. Функции}

Общие функции\+:
\begin{DoxyItemize}
\item Создание нового файла для растровых изображений;
\item Сохранение файла в любой момент времени в формате J\+P\+E\+G;
\item Открытие сохранённого файла для его изменения;
\item Вызов внешней пользовательской документации для изучения возможностей программы;
\end{DoxyItemize}

Функции обработки и создания изображений\+:
\begin{DoxyItemize}
\item Рисование линий произвольной формы попиксельно – т.\+н. инструмент “карандаш”;
\item Изменение масштаба рисунка – то есть, с помощью инструмента масштабирования возможность приблизиться к некоторой области для того, чтобы лучше её рассмотреть; размер конечного полотна при этом не изменяется ;
\item Выделение области на рисунке – область выделяется прямоугольная, с помощью специального инструмента, путем зажатия левой клавиши мыши и проведением ей вдоль рисунка до получения нужного размера области;
\item Изменение масштаба/поворот/отражение выделенной области на рисунке – так же возможно вырезать эту область (то есть на рисунке её не будет, но она будет в буфере);
\item Копирование выделенной области на рисунке (помещение выделенной области в буфер);
\item Заливка области, ограниченной линией произвольной формы – таким образом можно выбирать фон для изображения – заливкой чистой области;
\item Рисование форм таких, как\+: прямоугольник, эллипс (возможно, многоугольники);
\item Выбор различных цветов из цветовой палитры, представляющей собой набор квадратиков основных цветов;
\item Функция удаления рисуемого изображения попиксельно (ластик);
\item Изменение размера рисуемых линий/границ прямоугольника и эллипса/ластика;
\item Возможность создания полотна для творчества любого размера (пикс х пикс).
\end{DoxyItemize}

\paragraph*{4. Интерфейс}

Интерфейс будет представлять собой окно, в верхней части которого распложены кнопки общих функций (Сохранить, открыть, создать, справка), под ними – полоса элементов, относящихся к рисованию (толщина карандаша/ластика, палитра, выбор формы рисования – кривая, эллипс или прямоугольник, инструмент заливки, инструмент выделения, инструмент “ластик”, группа инструментов, относящаяся к изменению выделенной области (масштаб, поворот, отражение, удалить область), инструмент “масштаб” представляющий собой выдвижное окошко с процентами от исходного размера области) В самом верхнем правом углу – кнопки “закрыть”, “свернуть”, “развернуть”. Ещё ниже будет область для рисования белого (стандартного для фона) цвета – размер области зависит от выбора при создании файла. Стандартный цвет для фона – белый, для инструментов рисования – чёрный. Управление будет происходить с помощью мыши, выбирая нужный инструмент и рисуя, зажав левую клавишу и проводя по рабочей области. 